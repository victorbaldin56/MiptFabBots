% EXPLANATORY NOTE (explanatory_note.tex)
\documentclass[a4paper,12pt]{article}
\usepackage[T2A]{fontenc}
\usepackage[utf8]{inputenc}
\usepackage[russian]{babel}
\usepackage{hyperref}

\title{Пояснительная записка к проекту прыгающего робота}
\author{Балдин Виктор, Румянцев Иван, Голицын Артур}
\date{\today}

\begin{document}
\maketitle

\section{Обоснование выбора проекта}
Прыгающие роботы обладают преимуществами в преодолении сложного рельефа. Выбор ESP32 обусловлен:
\begin{itemize}
\item Наличием встроенного Wi-Fi
\item Низким энергопотреблением
\item Поддержкой Arduino IDE
\end{itemize}

\section{Описание системы}
\subsection{Аппаратная часть}
\begin{itemize}
\item ESP32 как основной контроллер
\item 2 сервопривода для управления прыжком
\item Литиевый аккумулятор 18650
\item Пружинный механизм из нерж. стали
\end{itemize}

\subsection{Программная часть}
\begin{itemize}
\item ESP32
\end{itemize}

\section{Процесс разработки}
\begin{enumerate}
\item Проектирование пружинного механизма в SolidWorks
\item Сборка прототипа
\item Калибровка сервоприводов
\item Оптимизация энергопотребления
\end{enumerate}

\section{Результаты}
\begin{itemize}
\item Дальность прыжка: 15-20 см
\item Время работы: 45 мин
\item Задержка управления: <100 мс
\item Репозиторий: \url{https://github.com/victorbaldin56/MiptFabBots}
\end{itemize}

\end{document}
