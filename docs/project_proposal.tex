% PROJECT PROPOSAL (project_proposal.tex)
\documentclass[a4paper,12pt]{article}
\usepackage[T2A]{fontenc}
\usepackage[utf8]{inputenc}
\usepackage[russian]{babel}
\usepackage{graphicx}

\title{Проектное предложение: проект BOTS}
\author{Балдин Виктор, Румянцев Иван, Голицын Артур}
\date{\today}

\begin{document}
\maketitle

\section{Аннотация}
Проект предполагает создание прототипа прыгающего робота на базе платформы Arduino с использованием микроконтроллера ESP32 для реализации дистанционного управления через Wi-Fi. Основная задача - разработка механической конструкции и системы управления, позволяющей осуществлять направленные прыжки по беспроводному каналу.

\section{Цель и задачи}
\textbf{Цель:} Создание управляемого по Wi-Fi прыгающего робота с автономным питанием.

\textbf{Задачи:}
\begin{itemize}
\item Разработка кинематической модели прыгающего механизма
\item Создание 3D-моделей деталей корпуса
\item Реализация системы управления на базе ESP32
\item Разработка мобильного приложения для управления
\item Тестирование и оптимизация энергопотребления
\end{itemize}

\section{Анализ аналогов}
\begin{itemize}
\item Salted Robotics Jumper (коммерческий) - пневматическая система прыжков
\item OpenSource JumpBot (DIY) - пружинный механизм с ИК-управлением
\item RoboBoi (образовательный) - ограниченная мобильность
\end{itemize}

\textbf{Отличия нашего проекта:}
Компактная конструкция с электромеханическим приводом, использование двусторонней Wi-Fi связи, открытая архитектура.

\section{Методология}
\begin{itemize}
\item 3D-проектирование в SolidWorks
\item Прототипирование на FDM-принтере
\item Программирование на Arduino
\item Использование протокола ESP32 для связи
\end{itemize}

\section{План работ}
\begin{tabular}{|l|l|}
\hline
Этап & Срок \\
\hline
Разработка механической части & 2 недели \\
Программирование контроллера & 1 неделя \\
Интеграция компонентов & 1 неделя \\
Тестирование и отладка & 2 недели \\
\hline
\end{tabular}

\end{document}